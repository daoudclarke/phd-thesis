%Bismillahi-r-Rahmani-r-Rahim
%\documentclass[12pt]{report}

%\ifx\thesishead\undefined
\newcommand*{\thesishead}{}

%Packages:
 \usepackage[round]{natbib}
 \usepackage{amsthm}
 \usepackage{amssymb}
 \usepackage[leqno]{amsmath}
 \usepackage{eufrak}
 \usepackage{graphs}
 %\usepackage{linguex}
 \usepackage{enumerate}
 \usepackage{algorithmic}
 \usepackage[plain]{algorithm}
 \usepackage{fancyvrb}
\usepackage{pstricks}
 \usepackage{subfigure}
\usepackage{enumerate}
\usepackage{bm}
\usepackage{graphicx}
%\usepackage{setspace}


 \newenvironment{itemizes}{\vspace{0.1cm}}{\\}
 \newcommand{\items}{\\ \vspace{0.1cm}\indent}

%For entailment
\newcommand{\swl}[1]{\rotatebox{90}{\parbox{4cm}{\begin{flushleft}\vspace{-0.1cm}{#1}\vspace{-0.1cm}\end{flushleft}}}}


%For meanings section
\DeclareMathOperator{\Ent}{Ent} 
\DeclareMathOperator{\Occ}{Occ}
 
\newcommand{\upa}{\Uparrow\!}
 
 %Commands for ontology section:
 %\newcommand{1}{\mathbf{1}}
%\newcommand{0}{\mathbf{0}}
\newcommand{\bv}{\mathbf{v}}
\newcommand{\bu}{\mathbf{u}}
\newcommand{\down}[1]{\left\downarrow(#1)\right.}
\newcommand{\downnb}[1]{\left\downarrow#1\right.}
\newcommand{\downe}[1]{\left\downarrow_\vdash(#1)\right.}
\newcommand{\up}[1]{\left\uparrow(#1)\right.}
%\newcommand{\com}{\mathrm{com}}
\DeclareMathOperator{\com}{com}
\DeclareMathOperator{\pre}{pre}
\DeclareMathOperator{\inc}{inc}
\DeclareMathOperator{\Cont}{Cont}
\DeclareMathOperator{\syn}{Syn}
\DeclareMathOperator{\Sub}{Sub}
\DeclareMathOperator{\FIS}{FIS}
%\DeclareMathSymbol{\Z}{\mathbin}{AMSb}{"5A}
%\DeclareMathSymbol{\R}{\mathbin}{AMSb}{"52}
\newcommand{\Par}{\mathrm{Par}}
\newcommand{\IC}{\mathit{IC}}

%Commands for definitions section:
 \newcommand{\R}{\mathbb{R}}
 \newcommand{\bra}[1]{\langle #1|}
\newcommand{\ket}[1]{|#1 \rangle}
\newcommand{\bracket}[2]{\langle #1| #2\rangle}
\newcommand{\inprod}[2]{\langle #1, #2 \rangle}
\newcommand{\Tr}[0]{\mathrm{Tr}}

\newcommand{\sbra}[1]{\langle #1\|}
\newcommand{\sket}[1]{\|#1 \rangle}
\newcommand{\sbracket}[2]{\langle #1\| #2\rangle}

%Theorems etc:
\newtheorem{prop}{Proposition}[chapter]
\newtheorem{assumption}{Assumption}
 \theoremstyle{definition}
 \newtheorem{defn}[prop]{Definition}
  \newtheorem{conj}[prop]{Conjecture}
% \theoremstyle{remark}
% \newtheorem*{remark}{Remark}
 \newtheorem{example}[prop]{Example}
\theoremstyle{remark}



\newcommand{\formulise}[1]{\begin{center}#1\\[6pt]\end{center}}


\fi





%\begin{document}


\section{Introduction to the Framework}

%In this chapter we define the abstract mathematical structure that will form the basis for the remainder of the thesis. This structure is based on the properties of the algebras that we can associate with a corpus as discussed in the last chapter. There we associated an element of an algebra with each word by looking at the contexts the word occurs in in the corpus, and we interpreted this element as the ``meaning'' of the word. Because the nature of the algebra is determined by the idea of meaning as context, we call implementations of the abstract structure we are about to describe ``context theories''.

In this chapter we define the context theoretic framework based on the theory of meaning as context developed in the previous chapter; the framework is formed from the central mathematical properties of the theory that were discussed there. These properties are derived from the assumption that the meaning of a string is purely determined by context; because of this, we can think of implementations of the framework as describing a theory about the contexts a string can occur in --- for this reason we call such implementations ``context theories''.

\subsection{Definition of Context Theory}

The important properties we wish to incorporate into the framework are as follows:
\begin{itemize}
\item Words and strings of words should be represented as vectors. We may wish to make use of techniques such as latent semantic analysis to derive vector representations of words; this ensures that such representations can be incorporated, but places the requirement that strings of words are also represented by vectors, based on our analysis in the previous chapter.
\item The vector space should in addition have a lattice structure. As we saw in the previous chapter, it is the lattice structure that informs us about entailment between strings, it is thus essential that this structure be incorporated into the framework.
\item The representation of the concatenation of strings can be viewed as a product of the representations of the individual strings for some distributive product (i.e.~the vector space forms an algebra). As we discussed in the previous chapter, this is a strong requirement to place on the mathematical structure. Imposing this structure is justified by the analysis of meaning as context and not only simplifies things from a mathematical perspective, but potentially opens up the vast amount of research available on these structures to be applied to computational linguistics.
\item There is a linear functional $\phi$ (the context-theoretic probability) on the vector space such that the lattice operations together with $\phi$ can be used to define an AL-space. This requirement ensures that $\phi$ behaves like a probability with respect to the lattice operations. This is important since the degree of entailment is defined in terms of $\phi$ and the lattice operations. We wish the degree of entailment to have the form of a conditional probability, and placing this requirement ensures that this will be the case for any implementation of the framework.
\item The algebra together with $\phi$ defines a non-commutative probability space. The benefits of this requirement are less clear at the moment, however there is evidence from our analysis in the previous chapter that imposing it is justified. In fact, this requirement will become more relevant later in the thesis when we discuss the representation of syntactic structure in the framework. Later we will hypothesise that the syntactic and semantic aspects of the representation of a word should combine \emph{freely} in the algebra. The study of non-commutative probability spaces has centred on this notion of \emph{free probability}, which is similar to the idea of independence in commutative variables.
\end{itemize}
We are able to combine these properties within the following definition:
\begin{defn}[Context Theory]
A context theory for an alphabet $A$ is a unital lattice-ordered algebra $\mathcal{A}$ together with a semigroup homomorphism from $A^*$ to $\mathcal{A}$, denoted $a \mapsto \hat{a}$ and a positive linear functional $\phi$ such that $\phi(\hat{\epsilon}) = 1$.
\end{defn}
In addition we shall also often require that the set $I = \{u : \phi(u) = 0\}$ is a sub-vector lattice of $\mathcal{A}$ --- that is, a subspace of $\mathcal{A}$ that is a vector lattice under the same partial ordering. We call a context theory that satisfies this condition a \emph{strong context theory}.

This definition incorporates all the properties we require:
\begin{itemize}
\item A string $x$ is represented by the vector $\hat{x}$; requiring this to be a semigroup homomorphism ensures that we can view strings as elements of an algebra.
\item We require that the algebra is lattice-ordered. While the lattice structure is essential, requiring a lattice-ordered algebra is a stronger requirement; this would be justified in our theory if the conjecture at the end of the last chapter is proven correct. We have made this requirement since in practice it is not a limitation: all the structures we will describe in the second half of this thesis are naturally lattice-ordered algebras.
\item requiring $\phi(\hat{\epsilon}) = 1$  makes $\mathcal{A}$ together with $\phi$ a non-commutative probability space;
\item we can define a norm on $\mathcal{A}$ based on $\phi$ that makes it an AL-space:
\end{itemize}
\begin{prop}[$\phi$-norm]
Given a context theory $\mathcal{A}$ with positive linear functional $\phi$ such that  $I = \{u : \phi(u) = 0\}$ is a sub-vector lattice of $\mathcal{A}$ we can define a norm $\|\cdot\|_\phi$ on $\mathcal{A}/I$ that defines an AL-space:
$$\|u\|_\phi = \phi(u^+) + \phi(u^-)$$
\end{prop}
\begin{proof}
The space $\mathcal{A}/I$ is the quotient space $\mathcal{A}/\equiv$ where $u\equiv v$ if $u,v \in I$, and is a vector lattice under the ordering of $\mathcal{A}$ \citep{Aliprantis:85}.\footnote{Effectively, it is the space formed by setting the subspace $I$ to zero.} The linear functional $\phi$ is well defined on this quotient space and satisfies $\phi(u) = 0$ if and only if $u$ is the zero of the quotient space. We need to show that $\|\cdot\|_\phi$ has the properties of a norm. For all $u \in \mathcal{A}/I$ we have:
\begin{itemize}
\item Positivity: $\|u\|_\phi \ge 0$ since $\phi$ is positive, and both $u^+$ and $u^-$ are positive.
\item Positive scalability: for $\alpha \in \R$, if $\alpha > 0$ then $\|\alpha u\|_\phi = \alpha\phi(u^+) + \alpha\phi(u^-)$. If $\alpha < 0$ then $(\alpha u)^+ = -\alpha u^-$ and $(\alpha u)^- = -\alpha u^+$ so $\|\alpha u\|_\phi = -\alpha\phi(u^+) - \alpha\phi(u^-)$. If $\alpha = 0$ then $\|\alpha u\|_\phi = 0$, thus for all $\alpha \in \R$, $\|\alpha u\|_\phi = |\alpha| \cdot \|u\|_\phi$.
\item Triangle inequality: we have $(u+v)^+ \le u^+ + v^+$ and $(u+v)^- \le u^- + v^-$. Then $\|u+v\|_\phi = \phi((u+v)^+) + \phi((u+v)^-) \le \phi(u^+ + v^+) + \phi(u^- + v^-) = \|u\|_\phi + \|v\|_\phi$.
\item Positive definiteness: it follows from $\phi(u) = 0$ if and only if $u = 0$ that $\|u\|_\phi = 0$ if and only if $u = 0$.
\end{itemize}
Finally, for $u,v \in \mathcal{A}$ with $u \ge 0$ and $v \ge 0$ we have $\|u + v\|_\phi = \phi(u+v) = \|u\|_\phi + \|v\|_\phi$ thus $\|\cdot\|_\phi$ defines an AL-space on $\mathcal{A}/I$.
\end{proof}

\subsection{Entailment}

The requirements that we placed on a context theory ensured that the space is a probability space in two ways. In particular, the definition of the degree of entailment that we defined previously in the form of a conditional probability applies equally well in the case of a context theory. We restate it here for the case of a context theory:
\begin{defn}[Degree of Entailment]
The degree of entailment $\Ent(x,y)$ between two strings $x$ and $y$ is defined as
$$\Ent(x,y) = \frac{\phi(\hat{x}\land\hat{y})}{\phi(\hat{x})}$$
\end{defn}
when $\phi(\hat{x}) \neq 0$, and is undefined otherwise

%\section{Overview of Part II}

%In the following chapter we review the task of recognising textual entailment, summarising different approaches to the problem, and identifying what we believe to be the main stumbling blocks encountered by these approaches. In the chapters that follow, we will try and show how our framework may ultimately enable us to overcome these problems. Specifically, we will show how uncertainty can be incorporated into traditional model-theoretic semantic representations using our framework; how ontologies can be related to vector space representations of words, and how this idea can be used to represent lexical ambiguity and how representations of the syntactic structure of language can be incorporated into the framework. Finally we will review free probability and demonstrate how syntactic and semantic representations can be combined in the framework, making the assumption that these two aspects combine freely.

%\bibliographystyle{plainnat}
%\bibliography{contexts.bib}

%\end{document} 


