%Bismillahi-r-Rahmani-r-Rahim
\documentclass{report}

\ifx\thesishead\undefined
\newcommand*{\thesishead}{}

%Packages:
 \usepackage[round]{natbib}
 \usepackage{amsthm}
 \usepackage{amssymb}
 \usepackage[leqno]{amsmath}
 \usepackage{eufrak}
 \usepackage{graphs}
 %\usepackage{linguex}
 \usepackage{enumerate}
 \usepackage{algorithmic}
 \usepackage[plain]{algorithm}
 \usepackage{fancyvrb}
\usepackage{pstricks}
 \usepackage{subfigure}
\usepackage{enumerate}
\usepackage{bm}
\usepackage{graphicx}
%\usepackage{setspace}


 \newenvironment{itemizes}{\vspace{0.1cm}}{\\}
 \newcommand{\items}{\\ \vspace{0.1cm}\indent}

%For entailment
\newcommand{\swl}[1]{\rotatebox{90}{\parbox{4cm}{\begin{flushleft}\vspace{-0.1cm}{#1}\vspace{-0.1cm}\end{flushleft}}}}


%For meanings section
\DeclareMathOperator{\Ent}{Ent} 
\DeclareMathOperator{\Occ}{Occ}
 
\newcommand{\upa}{\Uparrow\!}
 
 %Commands for ontology section:
 %\newcommand{1}{\mathbf{1}}
%\newcommand{0}{\mathbf{0}}
\newcommand{\bv}{\mathbf{v}}
\newcommand{\bu}{\mathbf{u}}
\newcommand{\down}[1]{\left\downarrow(#1)\right.}
\newcommand{\downnb}[1]{\left\downarrow#1\right.}
\newcommand{\downe}[1]{\left\downarrow_\vdash(#1)\right.}
\newcommand{\up}[1]{\left\uparrow(#1)\right.}
%\newcommand{\com}{\mathrm{com}}
\DeclareMathOperator{\com}{com}
\DeclareMathOperator{\pre}{pre}
\DeclareMathOperator{\inc}{inc}
\DeclareMathOperator{\Cont}{Cont}
\DeclareMathOperator{\syn}{Syn}
\DeclareMathOperator{\Sub}{Sub}
\DeclareMathOperator{\FIS}{FIS}
%\DeclareMathSymbol{\Z}{\mathbin}{AMSb}{"5A}
%\DeclareMathSymbol{\R}{\mathbin}{AMSb}{"52}
\newcommand{\Par}{\mathrm{Par}}
\newcommand{\IC}{\mathit{IC}}

%Commands for definitions section:
 \newcommand{\R}{\mathbb{R}}
 \newcommand{\bra}[1]{\langle #1|}
\newcommand{\ket}[1]{|#1 \rangle}
\newcommand{\bracket}[2]{\langle #1| #2\rangle}
\newcommand{\inprod}[2]{\langle #1, #2 \rangle}
\newcommand{\Tr}[0]{\mathrm{Tr}}

\newcommand{\sbra}[1]{\langle #1\|}
\newcommand{\sket}[1]{\|#1 \rangle}
\newcommand{\sbracket}[2]{\langle #1\| #2\rangle}

%Theorems etc:
\newtheorem{prop}{Proposition}[chapter]
\newtheorem{assumption}{Assumption}
 \theoremstyle{definition}
 \newtheorem{defn}[prop]{Definition}
  \newtheorem{conj}[prop]{Conjecture}
% \theoremstyle{remark}
% \newtheorem*{remark}{Remark}
 \newtheorem{example}[prop]{Example}
\theoremstyle{remark}



\newcommand{\formulise}[1]{\begin{center}#1\\[6pt]\end{center}}


\fi





\begin{document}

\section{Riesz Spaces and Positive Operators}

The previous two sections have described formalisms commonly used to describe meaning: broadly speaking, that of vector spaces and that of lattices. Until now, little attention within computational linguistics has been paid to how to combine these two areas. There is a large body of research within mathematical analysis into an area which merges the two formalisms: the study of \emph{partially ordered vector spaces}, \emph{vector lattices} (or \emph{Riesz spaces}), and \emph{Banach lattices}, and a special class of operators on these spaces called \emph{Positive operators}.

It is our belief that this area provides exciting new opportunities for combining the new, vector based representations, with old, ontological representations of meaning, and also provides a means of defining entailment between vector based representations.

The definitions and propositions of this section can be found in \cite{Abramovich:00} and \cite{Aliprantis:85}.

\begin{defn}[Partially ordered vector space]
A partially ordered vector space $V$ is a real vector space together with a partial ordering $\le$ such that:
\vspace{0.1cm}\\
\indent if $x \le y$ then $x + z \le y + z$\\
\indent if $x \le y$ then $\alpha x \le \alpha y$
\vspace{0.1cm}\\
for all $x,y,z \in V$. Such a partial ordering is called a \emph{vector space order} on $V$. If $\le$ defines a lattice on $V$ then the space is called a \emph{partially ordered vector space} or \emph{Riesz space}.

A vector $x$ in $V$ is called \emph{positive} if $x \ge 0$. The \emph{positive cone} of a partially ordered vector space $V$ is the set $V^+ = \{x \in V : x \ge 0\}$
\end{defn}

The positive cone has the following properties:
\begin{eqnarray*}
X^+ + X^+ \subseteq X^+\\
\alpha X^+ \subseteq X^+\\
X^+ \cap (-X^+) = \{0\}
\end{eqnarray*}
Any subset $C$ of $V$ satisfying the above three properties is called a \emph{cone} of $V$.

\begin{prop}
If $C$ is a cone in a real vector space $V$, then the relation $\le$ defined by $x \le y$ iff $y - x \in C$ is a vector space order on $V$, with $X^+ = C$.
\end{prop}

Part of our thesis is that meanings can be represented as positive elements of a vector space. If this is the case, then we can view words as operating on these meanings. Operators which map positive elements to positive elements are called \emph{positive}; there is a large body of work studying such operators.


\begin{defn}[Positive Operators]
An operator $A$ on a vector space $V$ is called \emph{positive} if $x \ge 0$ implies $Ax \ge 0$. It is called \emph{regular} if it can be denoted as the difference between two positive operators.
\end{defn}

Surprisingly, the set of regular operators on a vector space themselves form a vector lattice.

\begin{defn}[Lattice Homomorphism]
\end{defn}

\end{document} 