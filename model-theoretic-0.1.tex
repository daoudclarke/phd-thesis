%Bismillahi-r-Rahmani-r-Rahim
\documentclass[12pt]{report}

 \newcommand{\Cont}{\mathrm{Cont}}
\ifx\thesishead\undefined
\newcommand*{\thesishead}{}

%Packages:
 \usepackage[round]{natbib}
 \usepackage{amsthm}
 \usepackage{amssymb}
 \usepackage[leqno]{amsmath}
 \usepackage{eufrak}
 \usepackage{graphs}
 %\usepackage{linguex}
 \usepackage{enumerate}
 \usepackage{algorithmic}
 \usepackage[plain]{algorithm}
 \usepackage{fancyvrb}
\usepackage{pstricks}
 \usepackage{subfigure}
\usepackage{enumerate}
\usepackage{bm}
\usepackage{graphicx}
%\usepackage{setspace}


 \newenvironment{itemizes}{\vspace{0.1cm}}{\\}
 \newcommand{\items}{\\ \vspace{0.1cm}\indent}

%For entailment
\newcommand{\swl}[1]{\rotatebox{90}{\parbox{4cm}{\begin{flushleft}\vspace{-0.1cm}{#1}\vspace{-0.1cm}\end{flushleft}}}}


%For meanings section
\DeclareMathOperator{\Ent}{Ent} 
\DeclareMathOperator{\Occ}{Occ}
 
\newcommand{\upa}{\Uparrow\!}
 
 %Commands for ontology section:
 %\newcommand{1}{\mathbf{1}}
%\newcommand{0}{\mathbf{0}}
\newcommand{\bv}{\mathbf{v}}
\newcommand{\bu}{\mathbf{u}}
\newcommand{\down}[1]{\left\downarrow(#1)\right.}
\newcommand{\downnb}[1]{\left\downarrow#1\right.}
\newcommand{\downe}[1]{\left\downarrow_\vdash(#1)\right.}
\newcommand{\up}[1]{\left\uparrow(#1)\right.}
%\newcommand{\com}{\mathrm{com}}
\DeclareMathOperator{\com}{com}
\DeclareMathOperator{\pre}{pre}
\DeclareMathOperator{\inc}{inc}
\DeclareMathOperator{\Cont}{Cont}
\DeclareMathOperator{\syn}{Syn}
\DeclareMathOperator{\Sub}{Sub}
\DeclareMathOperator{\FIS}{FIS}
%\DeclareMathSymbol{\Z}{\mathbin}{AMSb}{"5A}
%\DeclareMathSymbol{\R}{\mathbin}{AMSb}{"52}
\newcommand{\Par}{\mathrm{Par}}
\newcommand{\IC}{\mathit{IC}}

%Commands for definitions section:
 \newcommand{\R}{\mathbb{R}}
 \newcommand{\bra}[1]{\langle #1|}
\newcommand{\ket}[1]{|#1 \rangle}
\newcommand{\bracket}[2]{\langle #1| #2\rangle}
\newcommand{\inprod}[2]{\langle #1, #2 \rangle}
\newcommand{\Tr}[0]{\mathrm{Tr}}

\newcommand{\sbra}[1]{\langle #1\|}
\newcommand{\sket}[1]{\|#1 \rangle}
\newcommand{\sbracket}[2]{\langle #1\| #2\rangle}

%Theorems etc:
\newtheorem{prop}{Proposition}[chapter]
\newtheorem{assumption}{Assumption}
 \theoremstyle{definition}
 \newtheorem{defn}[prop]{Definition}
  \newtheorem{conj}[prop]{Conjecture}
% \theoremstyle{remark}
% \newtheorem*{remark}{Remark}
 \newtheorem{example}[prop]{Example}
\theoremstyle{remark}



\newcommand{\formulise}[1]{\begin{center}#1\\[6pt]\end{center}}


\fi





\begin{document}

\chapter{Uncertainty in Model Theoretic Semantics}

In this chapter we examine context theoretic semantics as a way of incorporating information about uncertainty of meaning into model-theoretic representations of meaning. We will show that given a way of translating expressions into logical forms, we can define an algebra which represents the meaning equivalently. Given also a way of attaching probabilities to logical forms (which can be given a Bayesian interpretation), we can define a context theory allowing us to deduce degrees of entailment between expressions.

The context theoretic interpretation of ambiguity of meaning provides us with a recipe for representing ambiguity of words as well as syntactic ambiguity within the algebra, providing us with a principled way of reasoning with uncertainty of meaning within the model theoretic framework.

\section{From Logical Forms to Algebra}

Model-theoretic approaches generally deal with a subset of possible strings, the language under consideration, translating sequences in the language to a logical form, expressed in another, logical language. Relationships between logical forms are expressed by an entailment relation. This is summarised formally in the following definition:
\begin{defn}[Translation to Logical Form]
A \emph{translation to logical form} for some alphabet $A$ is a subset $L$ of $A^*$, together with a logical language $L' \subseteq B^*$ for some alphabet $B$ and a function $\lambda$ that assigns an element of $L'$ to each element of $L$ and a reflexive and transitive relation $\vdash$ on $L'$ called entailment.
\end{defn}

Given a translation to logical form for $A$, we can associate with each sequence in the language $L$ a projection on the space $\R^{L'}$ that expresses the information contained in the translation. Specifically,
\begin{prop}
Recall that for a subset $T$ of a set $S$, we denote the projection $P_T$ on $\R^S$ by 
$$P_T e_s = \left\{
\begin{array}{ll}
e_s & \text{if $s \in T$}\\
0 & \text{otherwise}
\end{array}
 \right.$$
 Where $e_s$ is the basis element of $\R^S$ corresponding to the element $s\in S$.
Given $u \in L'$ define a projection $P_u$ on $\R^{L'}$ by
$$P_u e_v = \left\{
\begin{array}{ll}
e_v & \text{if $v \vdash u$}\\
0 & \text{otherwise}
\end{array}
 \right.$$
 where $e_u$ is the basis element corresponding to $u \in L'$. For $x \in L$ define $P_x = P_{\lambda(x)}$ Then $P_x \le P_y$ if and only if $\lambda(x) \vdash \lambda(y)$.
\end{prop}
\begin{proof}
%Since $P_x$ and $P_y$ are band projections, they are commutative, and satisfy $P_x \le P_y$ if and only if $P_xP_y = P_x$.
%If $\lambda(x) \vdash \lambda(y)$ then
Clearly
$$P_xP_y e_u = \left\{
\begin{array}{ll}
e_u & \text{if $u \vdash \lambda(x)$ and $u \vdash \lambda(y)$}\\
0 & \text{otherwise}
\end{array},
 \right.$$
so if $\lambda(x) \vdash \lambda(y)$ then since $\vdash$ is transitive, if $u \vdash \lambda(x)$ then $u \vdash \lambda(y)$, so we must have $P_xP_y = P_x$. The projections $P_x$ and $P_y$ are band projections, so $P_xP_y = P_x$ if and only if $P_x \le P_y$.

Conversely, if $P_xP_y = P_x$ then it must be the case that $u \vdash \lambda(x)$ implies $u \vdash \lambda(y)$ for all $u \in L$, including $u = \lambda(x)$. Since $\vdash$ is reflexive, we have $\lambda(x) \vdash \lambda(x)$, so $\lambda(x) \vdash \lambda(y)$ which completes the proof.
\end{proof}

To help us understand this representation better, we wil show that it is closely connected to the ideal completion of partial orders. Define a relation $\equiv$ on $L'$ by $u \equiv v$ if and only if $u \vdash v$ and $v \vdash u$. Clearly $\equiv$ is an equivalence relation; we denote the equivalence class of $u$ by $[u]$. Equivalence classes are then partially ordered by $[u] \le [v]$ if and only if $u \vdash v$. Then it is straightforward to see that $P_x$ projects onto the space generated by the basis vectors corresponding to the elements $\bigcup \down{\lambda(x)}$.

\subsection{Application: Propositional Calculus}

Suppose we choose as our logical language $L'$ the language of a propositional calculus with the usual connectives $\lor$, $\land$ and $\lnot$, the logical constants $\top$ and $\bot$ representing ``true'' and ``false'' respectively, with $u \vdash v$ meaning ``infer $v$ from $u$'', behaving in the usual way. Then:
\begin{align*}
P_{u\land v} &= P_uP_v
	& P_{\lnot u} &= 1 - P_u + P_\bot\\
P_{u\lor v} &= P_u + P_v - P_uP_v
	& P_{\top} &=1
\end{align*}
%\begin{align}
%\tag{1} \label{and} P_{u\land v} &= P_uP_v\\
%\tag{2} \label{or} P_{u\lor v} &= P_u + P_v - P_uP_v\\
%\tag{3} \label{not} P_{\lnot u} &= 1 - P_u + P_\bot\\
%\tag{4} \label{true} P_{\top} &=1
%\end{align}

To see this, note that the equivalence classes of $\vdash$ form a Boolean algebra under the partial ordering induced by $\vdash$, with
\begin{align*}
[u\land v] & = [u] \land [v]
 & [u\lor v] & = [u] \lor [v]
  & [\lnot u] & = \lnot[u].
\end{align*}
Note that while the symbols $\land$, $\lor$ and $\lnot$ refer to logical operations on the left hand side, on the right hand side they are the operations of the Boolean algebra of equivalence classes; they are completely determined by $\vdash$.

Since the partial ordering carries over to the ideal completion we must have
\begin{align*}
\downnb{[u\land v]} &= \downnb{[u]} \cap \downnb{[v]}
& \downnb{[u\lor v]} &= \downnb{[u]} \cup \downnb{[v]}
\end{align*}
Since $u \vdash \top$ for all $u\in L'$, it must be the case that $\downnb{[\top]}$ contains all sets in the ideal completion. However the Boolean algebra of subsets in the ideal completion is larger than the Boolean algebra of equivalence classes; the latter is embedded as a Boolean sub-algebra of the former. Specifically, the least element in the completion is the empty set, whereas the least element in the equivalence class is represented as $\downnb{[\bot]}$. Thus negation carries over with respect to this least element:
$$\downnb{[\lnot u]} = (\downnb{[\top]} - \downnb{[u]})\cup \downnb{[\bot]}.$$

Since $\downnb{[\top]}$ contains all sets in the completion, $P_\top$ must project onto the whole space, that is $P_\top = 1$.

\section{Representing Uncertainty}

\subsection{Representing Bayesian Uncertainty}

The projection representation of translations to logical form allows us to associate an algebra (of projections) with the language $L$, however it does not quite give us a context theory for $L$. For that, we need a linear functional on the algebra of projections, and we will show how this can be done if we take a Bayesian approach to reasoning.


\subsection{Representing Lexical Ambiguity}

\subsection{Representing Syntactic Ambiguity}

\end{document} 