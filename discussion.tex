
\subsection{Discussion}

It is our belief that the recent developments in statistical techniques in computational linguistics require an accompanying development in our ideas of what meaning really is. Our idea of meaning is still connected to representations with their foundations in logic, and it is hard to imagine such representations capturing the subtleties of meaning that can be represented using the vector space representations acquired automatically.

What we propose then, is a theory of meaning that incorporates a vector space nature. Such a theory would be in better agreement with modern techniques in computational linguistics, and allow the more ``fine-grained'' descriptions of meaning that accompany them in vector representations.

The theory of meaning we propose has its foundations in the philosophy of Wittgenstein, who said ``Meaning just \emph{is} use'', and Firth, who said ``You shall know a word  by the company it keeps''. That is, the meaning of a sequence of symbols should be determined merely by looking at how they are used, or where they occur in a large amount of text.

The controversial part of what we propose is that meaning should be attached \emph{purely to context} without any assumed reference to the real world. It should be noted that this concept of meaning is quite different to our normal notion of meaning. An example that was proposed to me as an objection to such a definition of meaning is that we know that a \emph{man} is a \emph{human being} and a \emph{human being} is a \emph{mammal}. However, looking at the contexts that these words occur in, we would not expect significant overlap, for example, between the contexts that the word ``man'' and ``mammal'' occur in. There might be some overlap, but arguably not enough to demonstrate that a \emph{man} is a \emph{mammal}, that is that \emph{man} has all the properties of \emph{mammal}. We have to be able to accept, in our proposed definition of meaning that \emph{according to the definition} a man isn't a mammal! This is a consequence of the fact that we do not require our definition of meaning to be connected in any way to the real world, hence we call this view of meaning context-theoretic, as opposed to model-theoretic models of meaning.
%To separate this view of meaning from theories of meaning intended to represent the real world, we call the new interpretation of meaning \emph{context semantics}.\footnote{The term ``context semantics'' has already been used in the context of program analysis, as a less mathematical version of Girard's \emph{Geometry of Interaction} \citep[see for example][]{Girard:95}. Interestingly, the Geometry of Interaction makes use of Hilbert spaces and C* algebras for entirely different reasons to our own use of vector spaces. The use of the word ``context'' here is not related to our use of the term, so we think that no ambiguity will arise.} We may thus talk about the context semantics of a set of terms, in which it is not true that a \emph{man} is a \emph{mammal}, and their \emph{model-theoretic} semantics, in which this is true, as meanings are intended to be connected to a model of the real world.

Context-theoretic semantics places emphasis on the words themselves, and ignores the fact that words may represent concepts. This emphasis on context coincides with the modern use of statistical techniques in computational linguistics which have proved so effective in many applications. Whilst context-theoretic semantics may be dissatisfying from the perspective of some of our intuitive notions of meaning (a man is not a mammal), from a practical perspective context-theoretic semantics has the potential to successfully describe useful relationships in meaning. It is not often that it is practically \emph{useful} to know that a man is a mammal --- and the evidence from statistical techniques seem to show that the useful relationships very often \emph{are} those that are determined by context.

Of course this is not true in every situation; the context-theoretic approach may not be useful for many applications. It seems likely from what we have found that context-theoretic semantics would be good in situations where a limited form of reasoning is required, but in a ``fuzzy'' manner; and may be of limited use in situations which require complex reasoning and inference.

%\subsection{Predictions}
 
The framework makes several predictions about the nature of meaning, which, while they do not place severe limitations on the capabilities of the context-theoretic representation, do point in the direction of certain representations which are more suited to the context theoretic approach. One example is that the framework requires that meaning is \emph{associative}: if $a$, $b$ and $c$ are the representations of natural language expressions, then the product of $(ab)$ with $c$ is the same as the product of $a$ with $(bc)$; we believe this is a natural consequence of accepting that meaning is determined by context. However this would seem to limit the capabilities of the representation
 
% \section{A new philosophy of meaning}


 \section{Summary}
 